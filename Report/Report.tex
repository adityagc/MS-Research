\documentclass[a4paper,12pt]{article}
\title{Variable selection and estimation using the group lasso}
\date{}
\author{Aditya Chindhade}
\begin{document}
	\pagenumbering{gobble}
	\maketitle

	\newpage
	\tableofcontents
	
	\newpage
	\pagenumbering{arabic}
	\section{Sure.  I think one way to think about all this is the group lasso is also sometimes called the sum of norms.  Just think of two 	groups \theta^{(1)}, \theta^{(2)}, i.e., \theta = [\theta^{(1)}; \theta^{(2)}].  The group lasso penalty applied to \theta is	 		roughly ||\theta^{(1)}||_2 + ||\theta^{(2)}||_2, which is a sum of norms.  Since norms are nonnegative, you can think of this as the 		L1 norm applied to [||\theta^{(1)}||_2; ||\theta^{(2)}||_2] ... so again we can see that the group lasso is kind of like the lasso 		but 		working at the group level, i.e., enforcing sparsity at the group level.  Does that make sense?}
	
	\newpage
	\section{Introduction}
	\cite{yuan2006model}

	\newpage
	\section{Group Lasso}

	\newpage
	\section{Experiments}

	\newpage
	\section{Results}

	\newpage
	\section{Discussion}

	\newpage
	\section{Conclusion}
	
	\newpage
	\bibliography{Bibliography}
	\bibliographystyle{ieeetr}
\end{document}
